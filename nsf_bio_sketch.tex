% Upper-case    A B C D E F G H I J K L M N O P Q R S T U V W X Y Z
% Lower-case    a b c d e f g h i j k l m n o p q r s t u v w x y z
% Digits        0 1 2 3 4 5 6 7 8 9
% Exclamation   !           Double quote "          Hash (number) #
% Dollar        $           Percent      %          Ampersand     &
% Acute accent  '           Left paren   (          Right paren   )
% Asterisk      *           Plus         +          Comma         ,
% Minus         -           Point        .          Solidus       /
% Colon         :           Semicolon    ;          Less than     <
% Equals        =           Greater than >          Question mark ?
% At            @           Left bracket [          Backslash     \
% Right bracket ]           Circumflex   ^          Underscore    _
% Grave accent  `           Left brace   {          Vertical bar  |
% Right brace   }           Tilde        ~

%
% NSF Biographical Sketch: Dr. Henrik I Christensen
%
% Copyright (c) 2021 Henrik I Christensen
% All rights reserved.
%

\documentclass[svgnames,11pt]{article}
\usepackage[margin=1in]{geometry}
\usepackage[T1]{fontenc}
\usepackage{times}
\renewcommand\thesection{(\alph{section})}
\renewcommand\thesubsection{(\alph{subsection})}

\usepackage{calc}

\usepackage{lastpage}
%\newcommand{\pageoflastpage}{Page {\thepage} of \pageref*{LastPage}}
%\newcommand{\pageoflastpage}{{\thepage} of \pageref*{LastPage}}
\newcommand{\pageoflastpage}{}

% Redefine the plain pagestyle for the title page
\makeatletter
\let\oldps@plain\ps@plain
\renewcommand{\ps@plain}{\oldps@plain%
\renewcommand{\@evenfoot}{\hfil\pageoflastpage\hfil}%
\renewcommand{\@oddfoot}{\@evenfoot}}
\makeatother

% Use fancy for non-title pages
\usepackage{fancyhdr}
\fancyhead{}
\fancyfoot{}
\cfoot{\pageoflastpage}
\pagestyle{fancy}

\usepackage{xspace}

\usepackage[shortlabels]{enumitem}
\usepackage{graphicx}
\usepackage{varioref}
\usepackage[%
        colorlinks=true,urlcolor=,
        pdfpagelabels=true,hypertexnames=true,
        plainpages=false,naturalnames=true,
        ]{hyperref}
%
\usepackage{url}
\newcommand\doilink[1]{\href{http://dx.doi.org/#1}{#1}}
\newcommand\doi[1]{doi:\doilink{#1}}

\makeatletter
\newlength{\bibhang}
\setlength{\bibhang}{1em}
\newlength{\bibsep}
 {\@listi \global\bibsep\itemsep \global\advance\bibsep by\parsep}
\newlist{bibsection}{itemize}{3}
\setlist[bibsection]{label=,leftmargin={\bibhang+\widthof{[9]}},%
        itemindent=-\bibhang,
        itemsep=\bibsep,parsep=\z@,partopsep=0pt,
        topsep=0pt}
\newlist{bibenum}{enumerate}{3}
\setlist[bibenum]{label=\textbf{\arabic*.},leftmargin={\bibhang+\widthof{[999]}},%
        itemindent=-\bibhang,
        itemsep=\bibsep,parsep=\z@,partopsep=0pt,
        topsep=0pt}
\let\oldendbibenum\endbibenum
\def\endbibenum{\oldendbibenum\vspace{-.6\baselineskip}}
\let\oldendbibsection\endbibsection
\def\endbibsection{\oldendbibsection\vspace{-.6\baselineskip}}
\makeatother

\title{%
        \vspace{-2\baselineskip}
            \normalsize
            Biographical Sketch\\
            {\large\textbf{Dr.~Henrik~I.~Christensen}}\\
            \vspace{0.5\baselineskip}
            \hrule
            \vspace{0.5\baselineskip}
            Computer Science and Engineering, UC San Diego,\\
            e-mail: \href{mailto:hichristensen@ucsd.edu}{hichristensen@ucsd.edu},
            tel: +1-858-260-0570
        \vspace{-1.5ex}
        }
\date{}
\author{}

\hypersetup{%
        pdfsubject={NSF Biographical Sketch Henrik I.~Christensen},
	pdfauthor={Dr.~Henrik I.~Christensen},
        pdftitle={NSF Biographical Sketch of Dr.~Henrik I.~Christensen},
        pdfkeywords={}
        }

\newcommand{\ie}{i.e.\xspace}
\newcommand{\eg}{e.g.\xspace}

\begin{document}

\maketitle
\vspace{-2\baselineskip}

Dr.~Henrik I.~Christensen is the Qualcomm
Chancellor's Chair of Robot Systems and a Distinguished Professor of
Computer Science at Dept.~of Computer Science and Engineering UC San
Diego. He is also the director of the Institute for Contextual
Robotics. Prior to UC San Diego he was the founding director of
Institute for Robotics and Intelligent machines (IRIM) at Georgia
Institute of Technology (2006-2016). Dr.~Christensen does research on
systems integration, human-robot interaction, mapping and robot
vision. The research is performed within the Cognitive Robotics
Laboratory. He has published more than 350 contributions across AI,
robotics and vision. His research has a strong emphasis on "real
problems with real solutions". A problem needs a theoretical model,
implementation, evaluation, and translation to the real world. He is
actively engaged in the setup and coordination of robotics research in
the US (and worldwide). Dr.~Christensen received the Engelberger Award
2011, the highest honor awarded by the robotics industry. He was also
named the "Boeing Supplier of the Year 2011" while at Georgia Tech.
Dr.~Christensen is a fellow of American Association for Advancement of
Science (AAAS) and Institute of Electrical and Electronic Engineers
(IEEE). He received an honorary doctorate in engineering from Aalborg
University 2014.

\subsection{Professional Preparation}
 \begin{itemize}[label={\quad 9999--9999:},leftmargin=*,itemsep=0pt]
    \item[1981]Technical College of Frederikshavn, Frederikshavn, Denmark;
    Mechanical Engineering, Certificate;

    \item[1987] Aalborg University, Aalborg, Denmark;
    Electrical Engineering; M.~Sc.

    \item[1990] Aalborg University, Aalborg, Denmark;
    Electrical Engineering; Ph.D.
  \end{itemize}
  
\subsection{Appointments}

\begin{itemize}[label={\quad 9999--9999:},leftmargin=*,itemsep=0pt]
    \item[2019--present;]
        \textbf{Co-Founder}, Robust.AI, Palo Alto, CA
        \item[2016--present:] \textbf{Distinguished Professor},
        UC San Diego, La Jolla, CA

    \item[2006-2016:] \textbf{Distinguished Professor},
        Georgia Institute of Technology,
        Atlanta, GA

    \item[1996-2006:]
        \textbf{Professor/Director},
        Royal Institute of Technology,
        Stockholm, Sweden 

    \item[1996:]
        \textbf{Visiting Professor},
        University of Pennsylvania,
        Philadelphia, PA
        
    \item[1992-1996:]
      \textbf{Associate Professor},
        Aalborg University,
        Aalborg, Denmark
        
      \item[1990-1992:]
        \textbf{Research Associate}
        Aalborg University,
        Aalborg, Denmark

\end{itemize}

\subsection{Products}

\begin{bibenum}[itemsep=5pt]

\item Paz, D., Lai, P.-J., Harish, S., Zhang, H., Chan, N., Hu, C.,
  Binnani, S., and Christensen, H. Lessons learned from deploying
  autonomous vehicles at UC San Diego. In Field and Service Robotics
  (Tokyo, JP, August 2019).

\item Qui, Y. C., Pal, A., and Christensen, H. I. Target driven visual
  navigation exploiting object relationships. In The Conference on
  Robotic Learning (Boston, MA, November 2020).

\item  Paz, D., Zhang, H., Li, Q., Xiang, H., and Christensen, H. I.
  Probabilistic semantic mapping for urban autonomous driving
  applications. In International Conference on Intelligent Robots and
  Systems (IROS) (Las Vegas, NV, Oct 2020), IEEE/RSJ

\item Pal, A., Mondal, S., and Christensen, H. I. “looking at the
  right stuff” - guided semantic-gaze for autonomous driving. In
  Computer Vision and Pattern Recognition (CVPR) (Seattle, WA, June
  2020), IEEE/PAMI.

\item Pal, A., Nieto, C., and Christensen, H. I. DEDUCE: Diverse scEne
  Detection methods in Unseen Challenging Environments. In
  International Conference on Intelligent Robots and Systems (Macau,
  Oct 2019), IEEE/RSJ.
\end{bibenum}

\vspace{\baselineskip}

\noindent {\bf Other Significant Publications}
\vspace{\baselineskip}

\begin{bibenum}
\item Christensen, H. I., and Hager, G. Sensing and estimation. In
  Handbook of Robotics, B. Siciliano and O. Khatib, Eds. Springer
  Verlag, Berlin Heidelberg New York, May 2016, ch. 4
\item Christensen, H. I., and Nagel, H.-H., Cognitive Vision -
  Sampling the Spectrum. No. 3948 in Lecture Notes in Computer
  Science. Springer Verlag, Heidelberg, Apr. 2006
\item Large, E. W., Christensen, H. I., and Bajcsy, R. Scaling the
  dynamic approach to path planning and control: Competition among
  behavioral constraints. Intl. Jour. of Robotics Research 18, 1 (Jan.
  1999), 37-58.
\item Christensen, H.I., Kruijff, J, and Wyatt, J., Cognitive Systems,
  Sprinter Verlag, Heidelberg, April 2010.
\item Berman, F., Rutenbar, R. A., Hailpern, B., Christensen, H.,
  Davidson, S., Estrin, D., Franklin, M., Martonosi, M., Raghavan, P.,
  Stodden, V., and Szalay, A. S. Realizing the potential of data
  science. Communications of the ACM 61, 4 (Apr 2018), 67--72.

\end{bibenum}  
\subsection{Synergistic Activities}

\begin{bibenum}[itemsep=4pt]
  
   \item \textbf{Editorships:}
  
  (i) EIC: Foundations and Trends in Robotics (2008-2020)\\
  (ii) Board: IEEE PAMI, Intl. Jour of Robotics Research, Aut. Sys.,
       Field and Service Robotics, AI Magazine, MIT Press (Robotics)
      
  \item \textbf{University service:}
      
    (i) Director of Robotics Centers (KTH: 1996-2016), (GT: 2006-2016), (UCSD: 2016--)\\
    (ii) Coordinator of Robotics Education at UC San Diego (2018--)

    \item \textbf{Community service:}
      
      (i) Editor of the US National Robotics Roadmap - 2009, 2013, 2016 and 2020.\\
      (ii) Served on NSF CISE Advisory Board (2011-2014)\\
      (iii) Served as US FIRST FRC Judge and Regional Board ((2007--) GA and CA)  \\
      (iv)  Co-Chaired/Co-Program Chair for more than 35 IEEE conferences
    \item \textbf{Mentoring:}
      
      (i) Mentored more than 40 PhD students and 100+ M.Sc students\\
      (ii) Mentored 10+ undergraduate students to graduate admission\\
      (iii) Mentored 10+ High School student research 

\end{bibenum}


\end{document}

% Local Variables:
% End:
